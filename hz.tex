\documentclass{article}
\usepackage[resetfonts]{cmap}
\usepackage{lmodern}
\usepackage[czech]{babel}
\usepackage[utf8]{inputenc}
\usepackage[T1]{fontenc}
\usepackage[protrusion,expansion]{microtype}
\usepackage{multicol}
\usepackage{url}

\begin{document}
  \section{Expanze fontů}
  \begin{multicols}{2}
    \microtypesetup{protrusion=false, % Tato část ukázky bude sázena bez
                                      % zavěšování interpunkce.
                    expansion=false}  % Tento sloupec bude vysázen bez expanze
                                      % fontů.
    U~města Vrbna, obývaného převážně horníky, stával na Zámeckém vrchu hrad.
    Tady se udál za švédské vojny příběh velké lásky a nenávisti, jak jej
    vzpomíná stará pověst.  To bylo v~době, kdy na zdejším panství sídlili dva
    bratři, Hynek a Hanuš z~Vrbna. A~protože se stavěli proti císaři, dal je
    císař do klatby a jejich statky zabral. Hynek zavčasu uprchl do ciziny, ale
    Hanuš zaplatil své přesvědčení hlavou. Ještěže vdovu po Hanušovi a dcerušku
    Helenku nechali císařští na hradě!

    Když po letech přitáhli na Moravu a do Slezska Švédové, domácí obyvatelstvo
    se k~nim hlásilo přátelsky. Není se čemu divit, vždyť je spojovala společná
    nenávist k~císaři. To již Helenka vyrostla ve sličnou dívku, široko daleko
    proslulou svojí krásou. Když ji jednou zahlédl mladý švédský důstojník,
    zahořel k~ní láskou, a je třeba vyzradit, že také Helence se velice líbil.
    Švédové tehdy dobyli celou krajinu bruntálskou, ale z~vrbenského hradu
    císařské vypudit nedokázali. Hrad byl dobře zásoben střelivem a změnil se
    v~nedobytnou pevnost, kterou císařovi žoldáci zoufale bránili, aby uhájili
    vlastní životy. A~tak se zdálo, že milencům nebude dopřáno, aby se setkali.
    \columnbreak

    \microtypesetup{expansion=true} % Tento sloupec bude vysázen s expanzí.
    U~města Vrbna, obývaného převážně horníky, stával na Zámeckém vrchu hrad.
    Tady se udál za švédské vojny příběh velké lásky a nenávisti, jak jej
    vzpomíná stará pověst.  To bylo v~době, kdy na zdejším panství sídlili dva
    bratři, Hynek a Hanuš z~Vrbna. A~protože se stavěli proti císaři, dal je
    císař do klatby a jejich statky zabral. Hynek zavčasu uprchl do ciziny, ale
    Hanuš zaplatil své přesvědčení hlavou. Ještěže vdovu po Hanušovi a dcerušku
    Helenku nechali císařští na hradě!

    Když po letech přitáhli na Moravu a do Slezska Švédové, domácí obyvatelstvo
    se k~nim hlásilo přátelsky. Není se čemu divit, vždyť je spojovala společná
    nenávist k~císaři. To již Helenka vyrostla ve sličnou dívku, široko daleko
    proslulou svojí krásou. Když ji jednou zahlédl mladý švédský důstojník,
    zahořel k~ní láskou, a je třeba vyzradit, že také Helence se velice líbil.
    Švédové tehdy dobyli celou krajinu bruntálskou, ale z~vrbenského hradu
    císařské vypudit nedokázali. Hrad byl dobře zásoben střelivem a změnil se
    v~nedobytnou pevnost, kterou císařovi žoldáci zoufale bránili, aby uhájili
    vlastní životy. A~tak se zdálo, že milencům nebude dopřáno, aby se setkali.
  \end{multicols}
  
  \newpage
  \section{Zavěšování interpunkce}
  \begin{multicols}{2}
    \microtypesetup{expansion=false,  % Tato část ukázky bude sázena bez
                                      % expanze fontů.
                    protrusion=false} % Tento sloupec bude vysázen bez
                                      % zavěšování interpunkce.
    U~města Vrbna, obývaného převážně horníky, stával na Zámeckém vrchu hrad.
    Tady se udál za švédské vojny příběh velké lásky a nenávisti, jak jej
    vzpomíná stará pověst.  To bylo v~době, kdy na zdejším panství sídlili dva
    bratři, Hynek a Hanuš z~Vrbna. A~protože se stavěli proti císaři, dal je
    císař do klatby a jejich statky zabral. Hynek zavčasu uprchl do ciziny, ale
    Hanuš zaplatil své přesvědčení hlavou. Ještěže vdovu po Hanušovi a dcerušku
    Helenku nechali císařští na hradě!

    Když po letech přitáhli na Moravu a do Slezska Švédové, domácí obyvatelstvo
    se k~nim hlásilo přátelsky. Není se čemu divit, vždyť je spojovala společná
    nenávist k~císaři. To již Helenka vyrostla ve sličnou dívku, široko daleko
    proslulou svojí krásou. Když ji jednou zahlédl mladý švédský důstojník,
    zahořel k~ní láskou, a je třeba vyzradit, že také Helence se velice líbil.
    Švédové tehdy dobyli celou krajinu bruntálskou, ale z~vrbenského hradu
    císařské vypudit nedokázali. Hrad byl dobře zásoben střelivem a změnil se
    v~nedobytnou pevnost, kterou císařovi žoldáci zoufale bránili, aby uhájili
    vlastní životy. A~tak se zdálo, že milencům nebude dopřáno, aby se setkali.
    \columnbreak

    \microtypesetup{protrusion=true} % Tento sloupec bude vysázen se zavěšenou
                                     % interpunkcí.
    U~města Vrbna, obývaného převážně horníky, stával na Zámeckém vrchu hrad.
    Tady se udál za švédské vojny příběh velké lásky a nenávisti, jak jej
    vzpomíná stará pověst.  To bylo v~době, kdy na zdejším panství sídlili dva
    bratři, Hynek a Hanuš z~Vrbna. A~protože se stavěli proti císaři, dal je
    císař do klatby a jejich statky zabral. Hynek zavčasu uprchl do ciziny, ale
    Hanuš zaplatil své přesvědčení hlavou. Ještěže vdovu po Hanušovi a dcerušku
    Helenku nechali císařští na hradě!

    Když po letech přitáhli na Moravu a do Slezska Švédové, domácí obyvatelstvo
    se k~nim hlásilo přátelsky. Není se čemu divit, vždyť je spojovala společná
    nenávist k~císaři. To již Helenka vyrostla ve sličnou dívku, široko daleko
    proslulou svojí krásou. Když ji jednou zahlédl mladý švédský důstojník,
    zahořel k~ní láskou, a je třeba vyzradit, že také Helence se velice líbil.
    Švédové tehdy dobyli celou krajinu bruntálskou, ale z~vrbenského hradu
    císařské vypudit nedokázali. Hrad byl dobře zásoben střelivem a změnil se
    v~nedobytnou pevnost, kterou císařovi žoldáci zoufale bránili, aby uhájili
    vlastní životy. A~tak se zdálo, že milencům nebude dopřáno, aby se setkali.
  \end{multicols}

  \noindent Text byl převzat
  z~\url{http://www.abatar.cz/pohadky/bila_pani_vrbenska.htm}.

  \microtypesetup{expansion=false} % ... a ještě jednou bez expanze fontů.
  \noindent Text byl převzat
  z~\url{http://www.abatar.cz/pohadky/bila_pani_vrbenska.htm}.
  
  \microtypesetup{expansion=true,protrusion=false} % ... a ještě jednou bez
                                                   % zavěšování interpunkce.
  \noindent Text byl převzat
  z~\url{http://www.abatar.cz/pohadky/bila_pani_vrbenska.htm}.
\end{document}