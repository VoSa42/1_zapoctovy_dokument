\documentclass{article}
\usepackage[utf8]{inputenc}

\usepackage[resetfonts]{cmap}
\usepackage{lmodern}
\usepackage[czech]{babel}
\usepackage[T1]{fontenc}
\usepackage[protrusion,expansion]{microtype}

\setlength{\parindent}{10ex}

\begin{document}
\microtypesetup{protrusion=true, expansion=true}

\setlength{\parindent}{5ex}

\section*{Úvod}
Stereoskopie je dnes všeobecně známá spíše pod názvem „3D technologie“. Toto označení ale není zcela správné, 3D~=~3 dimenze (slovo dimenze znamená rozměr - původ z~latiny). 3D je tedy prostor, který má 3~rozměry. Takový prostor je všude kolem nás. Zatímco stereoskopie je systém, který umožňuje 2D obraz (tzn. plošný, např. na papíře nebo na monitoru) vnímat jako 3D obraz. V~principu jde tedy o~zrakovou iluzi.

\vspace{5mm} %5mm vertical space
\noindent\makebox[\linewidth]{\rule{\paperwidth}{0.4pt}}

\textit{Pozn.: druhá polovina 1.~zápočtového dokumentu je část práce soč, která je mým autorským dílem. Původně byla vypracována pomocí programu Microsoft Word, pro účely tohoto dokumentu jsem ji upravil a~vysázel v~\LaTeX u.}

\end{document}